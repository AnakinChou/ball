\documentclass[12pt,twoside,a4paper]{article}

\author{Andreas Kerzmann}
\title{PB - a Poisson-Boltzmann equation solver}

\usepackage{longtable}
\usepackage{float}
\restylefloat{figure}
\sloppy

\begin{document}

\maketitle

This document briefly describes the usage of the {\tt PB} application.

\section{Building and using {\tt PB}}

If you did not build the BALL library beforehand, read the tutorial and
install the BALL library first.

Change to the directory where you stored the BALL source files, say {\tt
/opt/BALL/source}. Then change to the directory of the {\tt PB}
application {\tt APPLICATIONS/PB}. Issue the {\tt make} command to build an
executable named {\tt PB}, which is the program this short introduction is
all about. In short:
\begin{verbatim}
cd /opt/BALL/source
cd APPLICATIONS/PB
make
\end{verbatim}

Now that the executable is built, we can start our first test run. Make sure
your {\tt LD\_LIBRARY\_PATH} environment varianble is set correctly. If you
do not know how to do this or what I am talking about, please refer to the
BALL tutorial. Run the program by typing {\tt ./PB} and pressing the
Enter-key. The output should look like figure \ref{helpmessage}.

\begin{figure}
\scriptsize
\begin{verbatim}
BALL -- Finite Difference Poisson Solver

PB [<options>]
   where <options> is one or more of the following possibilities:
     -P                   perform a Finite Difference Poisson calculation
     -A                   calculate the solvent accessible surface and
                            volume of the solute
     -E                   calculate the solvent excluded surface and
                            volume of the solute

further options:
     -p <FILE>            read <FILE> as a PDB file
     -h <FILE>            read <FILE> as a HyperChem file
     -H <FILE>            read <FILE> as a HyperChem file but do not assign
                            charges
     -o <FILE>            read FDPB options from <FILE>
     -c <FILE>            read charges from <FILE>
     -r <FILE>            read radii from <FILE>
     -t <FILE>            read charge and radius rules from <FILE>
     -u <FILE>            read charge rules from <FILE>
     -w <FILE>            read radius rules from <FILE>
     -0                   clear all charges in subsequently read structures
     -s                   calculate the solvation free energy by performing a 
                            second FDPB calculation in vacuum
     -n                   normalize all atom names in subsequently read
                            structures
     -b                   try to build the bonds (e.g. for PDB files)
     -d <FILE>            dump the atom charges, radii, and surface fractions
                            to <FILE> (for debugging)
     -v                   verbose output (implies ``verbosity 99'' in the
                            option file, print additional results and options)
     -x <RADIUS>          the probe sphere radius for solvent accessible and
                            solvent excluded surface calculations
                            [default = 1.4 A]
     -e <DIEL_CONST>      the dielectric constant of the surrounding medium
                            [default = 78.0]
     -f <DIEL_CONST>      the dielectric constant of interior of the solute
                            [default = 2.0]
     -i <IONIC_STRENGTH>  the ionic strength which will be used for the
                            Boltzmann part of the Poisson-Boltzmann equation
                            [default = 0.0 mol/l, i. e.  switched off]

  By default, charges and radii are taken from data/charges/PARSE.crg
  and data/radii/PARSE.siz.

  Charge and radius assignment options can be repeated. They are valid for all
  subsequently read structures.

\end{verbatim}
\normalsize
\caption{Help message of {\tt PB}}
\label{helpmessage}
\end{figure}

This usage message will always appear when starting the application without
giving any options. As you can see, many calculation options can be
controlled via the command line interface. For further options you can use
a so-called {\tt options} file containing more (and less frequently used)
options to tailor the calculation according to your needs. The usage of the
{\tt options} file will be briefly explained later.


\section{A simple expample}

Imagine you want to calculate the solvation free energy of a simple
monovalent spherical ion in pure water (\textit{i.~e.~} no salt solution
with counter ions), say K$^+$, which you have stored in a HyperChem HIN
file named {\tt k.hin}. Additionally you want to compare the calculated
value with the Born approximation of the solvation free energy.

To perform the calculation you need a charge and a radius for the ion. As
you stored the ion in a HyperChem file, its charge should already be
stored there. The radius has to be assigned by using another file
containing rules for assigning radii to atoms. For simplicity of the
example this data is stored in the file {\tt ions.rul}. See the BALL
reference manual for the usage of these rule files.

As you can see from the usage message (see figure \ref{helpmessage}) you
will need the following options for the calculation:
\begin{description}
\item[-P] perform a Poisson calculation
\item[-s] calculate the solvation free energy
\item[-w] read radius rule files from the file which is specified after
this option
\item[-v] be verbose (this is optional but try it once to see which
information you can get from this option)
\item[-h] use a HyperChem HIN file as input (specify the file name
immediately after this option)
\end{description}

Compiling all those options together will create the actual command which
you will have to type:
\begin{verbatim}
./PB -P -s -w ions.rul -h k.hin
\end{verbatim}
If the output looks somewhat similar to figure \ref{outputK} then you
successfully calculated your first solvation free energy with the BALL FDPB
solver, congratulations!

\begin{figure}[H]
\scriptsize
\begin{verbatim}
[18:29:16] Calculating the solvation free energy.
[18:29:16] first calculation step: solvent dielectric constant = 78.000000
[18:29:16] total energy:          28.8077 kJ/mol
[18:29:16] reaction field energy: -243.647 kJ/mol
[18:29:16] Calculating the solvation free energy.
[18:29:16] first calculation step: solvent dielectric constant = 78.000000
[18:29:16] second calculation step: solvent dielectric constant = 1.0 (vacuum)
[18:29:16] total energy:          383.598 kJ/mol
[18:29:16] reaction field energy: 99.2415 kJ/mol
[18:29:16] 
[18:29:16] Solvation energy as change of the total energy:   -354.79 kJ/mol
[18:29:16] Solvation energy as change of the reaction field: -342.889 kJ/mol
\end{verbatim}
\normalsize
\caption{Output of a simple ion solvation calculation}
\label{outputK}
\end{figure}

Now we check against the Born\cite{Born1920} formula (equation \ref{Born}) for
calculating the solvation free energy of spherical ions in water.

\begin{equation}
\Delta G_{\mathrm{solv}} = \frac{-z^2}{r_i} \left(1 - \frac{1}{\varepsilon_s}\right) \frac{e_0^2 N_A}{8 \pi \varepsilon_0}
\label{Born}
\end{equation}

Inserting our values for the K$^+$ ion (charge of 1, radius of 2.0\AA)
yields a solvation free energy of -342.80 kJ/mol which is in agreement with
our calculated values.

\section{The {\tt options} file}

The Poisson-Boltzmann solver of BALL knows many options. The most relevant
options will be introduced in this short introduction. Please consult the
BALL reference manula for documentation concerning further options. There
is a commented {\tt options} file {\tt PB.options} in the source directory
of the {\tt PB} application which contains all tunable options. This file
will also be in the appendix. 

The syntax of the options file is simple. It consists of lines which start
with an option keyword and contain the option value as the rest of the
line. Lines beginning with \verb+;+ or \verb+#+ will be ignored and can be
used for comments. Note that option keywords must not contain whitespaces,
{\textit i.~e.~}blanks, tabs and so on. Empty lines are also forbidden.

\begin{figure}[H]
\begin{verbatim}
; This is an example for the options syntax
;
; an imagenary option setting the number of hobbits
hobbit_count 4
\end{verbatim}
\end{figure}

\subsection{Options controlling grid properties}

\begin{description}
\item[spacing] is an option controlling the grid granularity. It defines
	the distance in \AA ngstrom (\AA) between every two grid points of the same
	axis-parallel line.
\item[border] sets the minimum distance in \AA between the bounding
	box of the molecule and the border of the grid. 
\end{description}

\subsection{Options controlling the solver}

\begin{description}
\item[boundary\_condition] sets the method for defining the boundary
	condition of the equation system. There are five different methods from
	which you can choose:
	\begin{description}
	\item[zero] simply sets potential to zero
	\item[Debye] uses the Debye approximation for initializing the potential
	\item[Coulomb] \emph{not implemented at the time of writing}
	\item[dipole] \emph{(default)} uses the effective dipole as boundary
		condition
	\item[focusing] computes a larger but less dense grid and takes its
		result for inititalizing the original grid. This method gives most
		accurate results but takes a lot more computing time.
	\end{description}
\item[charge\_distribution] controlles the distribution of molecule charges
	over grid points. You can choose between two algorithms.
	\begin{description}
	\item[uniform] \emph{(default)} uses a uniform distribution scheme for
		assigning charges to grid points
	\item[trilinear] distributes charges equally upon the eight closest grid
		points
	\end{description}
\item[dielectric\_smoothing] activates a smoothing algorithm for the
	dielectric constant. There are three options avilable.
	\begin{description}
	\item[none] switches dielectric smoothing off
	\item[uniform] uses a uniform smoothing function
	\item[harmonic] uses a harmonic smoothing function
	\end{description}
\end{description}

\subsection{Options controlling the molecular environment}

These options are also adaptable through command line switches

\begin{description}
\item[ionic\_strength] sets the ionic strength of the sourrounding medium.
	It has to be specified in mol/l. Command line switch: -i.
\item[solvent\_dielectric\_constant] defines the dielectric constant of the
	continuous solvent. Command line switch: -e.
\item[solute\_dielectric\_constant] defines the dielectric constant of the
	solute molecule(s). Command line switch: -f.
\end{description}


\newpage
\appendix

\section{The {\tt PB.options} file}

\scriptsize
\begin{verbatim}
; $Id: PB.tex,v 1.2 2002/12/12 09:53:11 oliver Exp $
; Example options file for the PB application
;
; all options are commented out
; just remove the ';' at the beginning of the options in order to use them
;
; set the verbosity level of the application 
; 0 produces very little output while 99 creates a lot of information
;verbosity 10 
;
; print the timing of all computationally expensive functions
;print_timing true
;
; set the grid spacing, i. e. the distance between every two grid points of
; the same axis-parralel line (in Angstrom)
;spacing 0.5
;
; set the border of the system, i. e. the space between the bounding box of
; the molecule(s) and the edge of the grid (given in Angstrom)
;border 5.0
;
; define the ionic strength of the solvent, i. e. activate the Boltzmann
; part of the calculation
; use units of mol/l
; this option can also be set by using the -i command line switch
;ionic_strength 0.15
;
; set the dielectric constant of the solute
; usually 2.0 is the value used in protein calculations
;solute_dielectric_constant 4
;
; set the solvent dielectric constant
; water has a dielectric constant of 78.5
;solvent_dielectric_constant 78.5
;
; set the probe radius for SAS calculations
; one usually assumes 1.4 Angstrom as the "radius" of water molecules
;probe_radius 1.4
;
; set the ion radius, i. e. the space around the molecule where ions from
; the solvent are exluded (in Angstrom)
;ion_radius 2.0
;
; set the temperature of the system (in Kelvin)
; the default is 298.15 Kelvin
;temperature 290
;
; define the boundary condition for inititalizing the equation solver
; there are several options: zero, Debye, Coulomb, dipole and focusing
;boundary_condition focusing
;
; choose the algorithm for distributing the atom charges over the grid
; there are two methods: uniform and trilinear
;charge_distribution uniform
;
; choose the method of smoothing the dielectric constant
; there are three methods: none, uniform and harmonic
; note that reaction field energies cannot be computed if dielectric
; smoothing is activated
;dielectric_smoothing none
;
; define a spatial offset for the calculation
; you can displace the whole grid by defining this offset vector
;offset (1.0 1.0 1.0)
;
; set the rms criterion for termination of the solver iterations
; if the root mean sqaure deviation of the equation system is below this
; value and the maximum resiual is below the max_criterion limit the
; equation system did converge
;rms_criterion 1e-6
;
; set the maximum residuals criterion for the solver
; if the root mean sqaure deviation of the equation system is below this
; value and the maximum resiual is below the max_criterion limit the
; equation system did converge
;max_criterion 1e-6
;
; set the frequencz of checking the convergence criterions
; use this options for defining the number of iterations that will be
; performed before checking the criterions again.
;check_after_iterations 5
;
; set the maximum number of iterations
; if the solver did not converge until thid number of iterations stop the
; solver
;max_iterations 1000
;
; set the lower corner of the grid
;lower (0.0 0.0 0.0)
;
; set the upper corner of the grid
;upper (0.0 0.0 0.0)
;
; set the lower corner of the bounding box of the molecule(s)
;bounding_box_lower (0.0 0.0 0.0)
;
; set the upper corner of the bounding box of the molecule(s)
;bounding_box_upper (0.0 0.0 0.0)
\end{verbatim}
\normalsize

\end{document}
