\documentclass[12pt,twoside,a4paper]{article}

\author{Andreas Kerzmann}
\title{PB - a Poisson-Boltzmann equation solver}

\usepackage{longtable}
\sloppy

\begin{document}

This document briefly describes the usage of the {\tt PB} application.

\section{Building and using {\tt PB}}

If you did not build the BALL library beforehand, read the tutorial and
install the BALL library first.

Change to the directory where you stored the BALL source files, say {\tt
/opt/BALL/source}. Then change to the directory of the {\tt PB}
application {\tt APPLICATIONS/PB}. Issue the {\tt make} command to build an
executable named {\tt PB}, which is the program this short introduction is
all about. In short:
\begin{verbatim}
cd /opt/BALL/source
cd APPLICATIONS/PB
make
\end{verbatim}

Now that the executable is built, we can try our first test run. Make sure
your {\tt LD\_LIBRARY\_PATH} environment varianble is set correctly. If you
do not know how to do this or what I am talking about, please refer to the
BALL tutorial. Run the program by typing {\tt ./PB} and pressing the
Enter-key. The output should look like figure \ref{helpmessage}.

\begin{figure}
\scriptsize
\begin{verbatim}
BALL -- Finite Difference Poisson Solver

PB [<options>]
   where <options> is one or more of the following possibilities:
     -P                   perform a Finite Difference Poisson calculation
     -A                   calculate the solvent accessible surface and
                            volume of the solute
     -E                   calculate the solvent excluded surface and
                            volume of the solute

further options:
     -p <FILE>            read <FILE> as a PDB file
     -h <FILE>            read <FILE> as a HyperChem file
     -H <FILE>            read <FILE> as a HyperChem file but do not assign
                            charges
     -o <FILE>            read FDPB options from <FILE>
     -c <FILE>            read charges from <FILE>
     -r <FILE>            read radii from <FILE>
     -t <FILE>            read charge and radius rules from <FILE>
     -u <FILE>            read charge rules from <FILE>
     -w <FILE>            read radius rules from <FILE>
     -0                   clear all charges in subsequently read structures
     -s                   calculate the solvation free energy by performing a 
                            second FDPB calculation in vacuum
     -n                   normalize all atom names in subsequently read
                            structures
     -b                   try to build the bonds (e.g. for PDB files)
     -d <FILE>            dump the atom charges, radii, and surface fractions
                            to <FILE> (for debugging)
     -v                   verbose output (implies ``verbosity 99'' in the
                            option file, print additional results and options)
     -x <RADIUS>          the probe sphere radius for solvent accessible and
                            solvent excluded surface calculations
                            [default = 1.4 A]
     -e <DIEL_CONST>      the dielectric constant of the surrounding medium
                            [default = 78.0]
     -f <DIEL_CONST>      the dielectric constant of interior of the solute
                            [default = 2.0]
     -i <IONIC_STRENGTH>  the ionic strength which will be used for the
                            Boltzmann part of the Poisson-Boltzmann equation
                            [default = 0.0 mol/l, i. e.  switched off]

  By default, charges and radii are taken from data/charges/PARSE.crg
  and data/radii/PARSE.siz.

  Charge and radius assignment options can be repeated. They are valid for all
  subsequently read structures.

\end{verbatim}
\normalsize
\caption{Help message of {\tt PB}}
\label{helpmessage}
\end{figure}

This usage message will always appear when starting the application without
giving any options. As you can see, many calculation options can be
controlled via the command line interface. For further options you can use
a so-called {\tt options} file containing more (and less fequently used)
options to tailor the calculation according to your needs. The usage of the
{\tt options} file will be briefly explained later.


\section{A simple expample}

Imagine you want to calculate the solvation free energy of simple
monovalent spherical ion in pure water (\textit{i.~e.~} no salt solution
with counter ions), say $K^+$, which you have stored in a HyperChem HIN
file named {\tt k.hin}. Additionally you want to compare the calculated
value with the Born approximation of the solvation free energy.

To perform the calculation you need a charge and a radius for the ion. As
you stored the ion in a HyperChem file, this charge should already be
stored there. The radius has to be assigned by using another file
containing a rule for assigning radii to atoms. For simplicity of the
example this data is stored in the file {\tt ions.rul}. See the BALL
reference manual for the usage of these rule files.

As you can see from the usage message (see figure \ref{helpmessage}) you
will need the following options for the calculation:
\begin{description}
\item[-P] perform a Poisson calculation
\item[-s] calculate the solvation free energy
\item[-w] read radius rule files from the file which is specified after
this option
\item[-v] be verbose (this is optional but try it once to see which
information you can get from this option)
\item[-h] use a HyperChem HIN file as input (specify the file name
immediately after this option)
\end{description}

Compiling all those options together will create the actual command which
you will have to type:
\begin{verbatim}
./PB -P -s -w ions.rul -h k.hin
\end{verbatim}
If the output looks somewhat similar to figure \ref{outputK} then you
successfully calculated your first solvation free energy with the BALL FDPB
solver, congratulations!

\begin{figure}
\scriptsize
\begin{verbatim}
[18:29:16] Calculating the solvation free energy.
[18:29:16] first calculation step: solvent dielectric constant = 78.000000
[18:29:16] phi
[18:29:16] 
[18:29:16] 0x1eb750total energy:          28.8077 kJ/mol
[18:29:16] reaction field energy: -243.647 kJ/mol
[18:29:16] Calculating the solvation free energy.
[18:29:16] first calculation step: solvent dielectric constant = 78.000000
[18:29:16] second calculation step: solvent dielectric constant = 1.0 (vacuum)
[18:29:16] phi
[18:29:16] 
[18:29:16] 0x1eb750total energy:          383.598 kJ/mol
[18:29:16] reaction field energy: 99.2415 kJ/mol
[18:29:16] 
[18:29:16] Solvation energy as change of the total energy:   -354.79 kJ/mol
[18:29:16] Solvation energy as change of the reaction field: -342.889 kJ/mol
\end{verbatim}
\normalsize
\caption{Output of a simple ion solvation calculation}
\label{outputK}
\end{figure}

Now we check against the Born\cite{Born1920} formula (equation \ref{Born}) for
calculating the solvation free energy of spherical ions in water.

\begin{equation}
\Delta G_{\mathrm{solv}} = \frac{-z^2}{r_i} \left(1 - \frac{1}{\varepsilon_s}\right) \frac{e_0^2 N_A}{8 \pi \varepsilon_0}
\label{Born}
\end{equation}

Inserting our values for the $K^+$ ion (charge of 1, radius of 2.0$\AA$)
yields a solvation free energy of 342.80 kJ/mol which is in agreement with our ccalculated values.

\end{document}
