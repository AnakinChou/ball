%-----------------------------------------
% used packages
%-----------------------------------------
\usepackage{a4}
% use A4 paper

%\usepackage[draft]{graphicx}
\usepackage{graphicx}
% use graphics package

\usepackage{times}
% use postscript fonts

\usepackage{psboxit}
\newcommand{\graybox}[1]{\psboxit{box .85 setgray fill}{\fbox{#1}}}
% display gray shaded boxes using postscript commands
%


%-----------------------------------------
% quote environment for quatations
%-----------------------------------------
% in part stolen from LaTeX companion
\usepackage{ifthen}
\newsavebox{\QuoteNameBox}
\newenvironment{Quote}[1]%
	{%
		\sbox{\QuoteNameBox}{{\it #1}}%
		\begin{list}{}{%
			\setlength{\rightmargin}{\leftmargin}}%
				\item[]``\ignorespaces}%
	{\unskip''\hfill \usebox{\QuoteNameBox}\end{list}}


\usepackage{changebar}
% This package defines the two commands \cbstart and \cbend. It then displays a
% bar between the start and the end command

\usepackage{color}
% This package allows the coloring of text (e.g. in an ipe minipage)

\usepackage{float}
% the float package allows a better placement of all floats (figures,tables) or
% even new floats it allows the following kinds of boxes:
% \shadowbox, \doublebox, \ovalbox, \Ovalbox

\makeatletter
	\renewcommand{\@makecaption}[2]{
		\vspace{3pt}
		\noindent{\color{blue}\rule{\textwidth}{1pt}}\par
		\noindent{\small{\sffamily #1:}\hspace{5pt}\itshape #2\par}
		\vspace{5pt}
	}
\makeatother
% 
%  create a fancy float caption: 
%    - blue rule under the float
%    - the figure text set in sans serif
%    - the caption text in italics

\setcounter{topnumber}{1}
\setcounter{bottomnumber}{1}
%
%  	some settings for the floats: just on float at the
%		top of a page and on at the bottom

	
%-----------------------------------------
% Font definitions 
%-----------------------------------------

\usepackage{amsfonts}
\usepackage{amsmath}
\usepackage{amsthm}
\usepackage{amssymb}

%-----------------------------------------
% Textlayout 
%-----------------------------------------

\newcommand{\clearemptydoublepage}{\newpage{\pagestyle{empty}\cleardoublepage}}

%\usepackage{a4wide}
\setlength{\textheight}{21cm}
\setlength{\textwidth}{14.5cm}
\setlength{\oddsidemargin}{10mm}
\setlength{\evensidemargin}{2mm}
\setlength{\topmargin}{0mm}
%\setlength{\marginparsep}{5mm}
%\setlength{\marginparwidth}{2cm}
%\renewcommand{\baselinestretch}{1.21}
\large\normalsize

%-----------------------------------------
% some shorthands
%-----------------------------------------

\providecommand{\R}{\mathbb R}
\providecommand{\Q}{\mathbb Q}
\providecommand{\N}{\mathbb N}
\providecommand{\Z}{\mathbb Z}

%-----------------------------------------
% redefinitions of sectioning comands
%-----------------------------------------

\usepackage{fancyheadings}
% allows you an easy adaption of the headings

\usepackage{fancybox}
% you can use shaded boxes e.g. for figures it allows the following kinds of boxes:
% \shadowbox, \doublebox, \ovalbox, \Ovalbox

\addtolength{\headrulewidth}{0.2pt}
\addtolength{\footrulewidth}{0.6pt}
\addtolength{\headwidth}{1cm}
\addtolength{\footskip}{10mm}
\addtolength{\headsep}{5mm}
\addtocounter{secnumdepth}{2}
\setcounter{tocdepth}{3}
\setcounter{secnumdepth}{3}

\renewcommand{\chaptermark}[1]{\hfill\markboth{#1}{}\hfill}
\renewcommand{\sectionmark}[1]{\hfill\markright{\thesection\ #1}\hfill}
\lhead{}
\rhead{}
\chead[\fancyplain{}{\textrm{\leftmark}}]%
      {\fancyplain{}{\textrm{\rightmark}}}
\cfoot{}
\rfoot[\fancyplain{}{}]{\fancyplain{}{\bfseries\thepage}}
\lfoot[\fancyplain{}{\bfseries\thepage}]{\fancyplain{}{}}


% the \chapter ...
\makeatletter
%\renewcommand\thesection 			{{\sffamily\thechapter.\@arabic\c@section}}
%\renewcommand\thesubsection   {{\sffamily\thesection.\@arabic\c@subsection}}
%\renewcommand\thesubsubsection{{\sffamily\thesubsection.\@arabic\c@subsubsection}}
\renewcommand{\@chapapp}{\sffamily}
%\renewcommand\section{\@startsection {section}{1}{\z@}%
%	{-3.5ex \@plus -1ex \@minus -.2ex}%
%	{2.3ex\@plus.2ex}%
%	{\normalfont\Large\sffamily\bfseries}
%}
%\renewcommand\subsection{\@startsection{subsection}{2}{\z@}%
%	{-3.25ex\@plus-1ex \@minus -.2ex}%
%	{1.5ex \@plus .2ex}%
%	{\normalfont\large\sffamily\bfseries}
%}
%\renewcommand\subsubsection{\@startsection{subsubsection}{3}{\z@}%
%	{-3.25ex\@plus-1ex \@minus -.2ex}%
%	{1.5ex \@plus .2ex}%
%	{\normalfont\normalsize\sffamily\bfseries}
%}
\makeatother

%-----------------------------------------
% macro for missing pieces
%-----------------------------------------
\usepackage{changebar}
\newcommand{\missing}[1]{
	\cbstart  % add a gray bar at the side of the page
	\message{^^JMISSING: #1^^J}
	\par\noindent\fbox{{\bf MISSING:} #1}\par
	\cbend
}

%-----------------------------------------
% rules for figures
%-----------------------------------------

\topfigrule
\botfigrule
\setlength{\shadowsize}{2pt}

%-----------------------------------------
% some shorthands
%-----------------------------------------

\usepackage{xspace}
% you should use \xspace in defining shorthands 
% e.g. \newcommand{\eg}{e.g., \xspace} \@ is for the correct spacing after punctuation

\newcommand{\eg}{{\it e.g.}\xspace}
\newcommand{\ie}{{\it i.e.}\xspace}
\newcommand{\ea}{{\it et al.}\xspace}
\newcommand{\etc}{{\it etc.}\xspace}
\def\CPP{C\raise.08ex\hbox{\tt ++}\xspace}


%-----------------------------------------
% the index
%-----------------------------------------
\usepackage{makeidx}

\newcommand{\Index}[1]{#1\index{#1}}
% prints the term and creates an index entry

\newcommand{\class}[1]{{\ttfamily{#1}}\index{#1@{\ttfamily{#1}}
(BALL~class)}\index{BALL!classes!#1@{\ttfamily{#1}}}}
% BALL class names are typeset in typewriter bold and indexed automatically:
%  - first with their class name
%  - then under BALL!classes!<classname>

\newcommand{\STLclass}[1]{{\ttfamily{#1}}\index{#1@{\ttfamily{#1}}
(STL~class)}\index{STL!#1 class@{\ttfamily{#1}}}}
% STL class names are typeset in typewriter bold and indexed automatically:
%  - first with their class name
%  - then under STL!<classname> class

\newcommand{\function}[1]{{\ttfamily{#1}}\index{#1@{\ttfamily{#1}}
(BALL~function)}\index{BALL!functions!#1@{\ttfamily{#1}}}}
% BALL function names are typeset in typewriter bold and indexed automatically:
%  - first with their name
%  - then under BALL!functions!<function name>

\newcommand{\method}[1]{{\ttfamily{#1}}\index{#1@{\ttfamily{#1}}
(method)}\index{BALL!functions!#1@{\ttfamily{#1}}}}
% method names are typeset in typewriter bold and indexed automatically:
%  - first with their name
%  - then under BALL!methods!<function name>

\newcommand{\member}[2]{%
	{\ttfamily{#2}}%
	\index{#1@{\ttfamily{#2}}(member of \ttfamily{#1})}%
	\index{BALL!#1!#2@{\ttfamily{#1}}}%
}%
% method names are typeset in typewriter bold and indexed automatically:
%  - first with their name
%  - then under BALL!members!<function name>

\newcommand{\attribute}[2]{{\ttfamily{#2}}\index{#2@{\ttfamily{#2}}
({\ttfamily #1} attribute)}\index{BALL!!#1!#2@{\ttfamily{#2}}}}
% attribute names are typeset in typewriter bold and indexed automatically:
%  - first with their name and (<class> attribute)
%  - then under BALL!<class>!<attribute>

\newcommand{\macro}[1]{{\ttfamily{#1}}\index{#1@{\ttfamily{#1}}
(BALL~macro)}\index{BALL!macros!#1@{\ttfamily{#1}}}}
% BALL macros names are typeset in typewriter bold and indexed automatically:
%  - first with their name
%  - then under BALL!macros!<macro name>

\newcommand{\namespace}[1]{{\ttfamily{#1}}\index{#1@{\ttfamily{#1}}
(BALL~namespace)}\index{BALL!namespaces!#1@{\ttfamily{#1}}}}
% BALL macros names are typeset in typewriter bold and indexed automatically:
%  - first with their name
%  - then under BALL!namespaces!<name>

\newcommand{\type}[1]{{\ttfamily{#1}}\index{#1@{\ttfamily{#1}}
(BALL~type)}\index{BALL!types!#1@{\ttfamily{#1}}}}
% BALL macros names are typeset in typewriter bold and indexed automatically:
%  - first with their name
%  - then under BALL!types!<macro name>

\newcommand{\file}[1]{%
	{\ttfamily\mbox{#1}}%
	\index{#1@{\ttfamily{#1}} (file)}%
}%
% BALL file names are typeset in typewriter bold and indexed automatically:
% with their name and "(file)" appended

\newcommand{\directory}[1]{%
	{\ttfamily\mbox{#1}}%
	\index{#1@{\ttfamily{#1}} (directory)}%
}%
% BALL directories are typeset in typewriter bold and indexed automatically:
% with their name and "(directory)" appended

\newcommand{\option}[1]{%
	{\ttfamily\mbox{#1}}%
	\index{configure!options!#1@{\ttfamily\mbox{#1}}}%
}%
% BALL file names are typeset in typewriter bold and indexed automatically:
% as configure/options/<name>"

\newcommand{\exception}[1]{{\ttfamily{#1}}\index{#1@{\ttfamily{#1}}
(BALL~class)}\index{BALL!exceptions!#1@{\ttfamily{#1}}}}
% BALL exceptions names are typeset in typewriter bold and indexed automatically:
%  - first with their name
%  - then under BALL!exceptions!<exception name>

\newcommand{\newtermdef}[1]{{\em #1}\index{#1!definition}}
% prints the term in italics and includes creates an index entry
% with the subentry "definition"

\newcommand{\newterm}[1]{{\em #1}\index{#1}}
% prints the term in italics and creates an index entry

\newcommand{\URL}[1]{{\bfseries\ttfamily\mbox{#1}}}
\newcommand{\eMail}[1]{{\bfseries\ttfamily\mbox{#1}}}

%--------------------------------------------
%  the style of embedded listings
%--------------------------------------------
\usepackage{listings}
% formatting of the listings
\lstset{
	%% we use ANSI C++
	language=[ANSI]C++,
%
	%% our tabs are always expanded to 2 blanks
	tabsize=2,
%
	%% the font size of the listing
	basicstyle=\footnotesize\ttfamily,
%
	%% make the listing as wide as \textwidth
	%spread=-0.7cm,
%
	%% draw a line at the left of the listing
	frame=L,
%
	%% captions appear at the bottom of the listing
	%captionpos=b,
%
	%% and make the corners of the frame round
	frameround=ffff,
%
	%% the label font
	%labelstyle=\sffamily,
%
	%% line numbers for each line
	%labelstep=0
}
\newcommand{\linelisting}{\lstinline[frame=,basicstyle=\small]}


%%
%%
%%     FAQ macros
%%
\newcounter{FAQcounter}
\newcounter{FAQseparatorCounter}
\setcounter{FAQcounter}{0}
\setcounter{FAQseparatorCounter}{0}
\newcommand{\FAQquestion}[1]{%
	\ifthenelse{\theFAQseparatorCounter=0}{}{\vspace{0.5mm}\begin{center}\rule{0.5\textwidth}{0.1pt}\end{center}\vspace{0.5mm}}%
	\stepcounter{FAQcounter}%
	\stepcounter{FAQseparatorCounter}%
	\noindent{\normalfont\bf\sffamily Question \theFAQcounter:} {\em #1 \par}%
}
\newcommand{\FAQanswer}[1]{\bigskip\noindent{\normalfont\bf\sffamily Answer:} #1\par}	
\newcommand{\FAQURL}[1]{\mbox{\tt #1}}
\newcommand{\FAQCategory}[1]{
	\par\vspace{0.5cm}
	\graybox{\hspace{-3pt}\makebox[\linewidth][c]{\normalfont\bf\sffamily\Large{#1}}}
	\par\vspace{0.6cm}
	\setcounter{FAQseparatorCounter}{0}
}
