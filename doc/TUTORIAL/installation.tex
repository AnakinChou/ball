\section{System Requirements}

\paragraph{Compiler}
\index{compiler}
  BALL requires an ANSI compliant \CPP compiler or GNU gcc
  (version above 2.95 or above).
  It has been successfully built and tested on the following platforms:
	\begin{itemize}	
    \item Linux 2.0 using g++ 2.95.1 and 2.95.2
    \item Solaris 2.6 and 7/SPARC using g++ 2.95.2
    \item Solaris 7/SPARC using Kuck \& Associate (KAI) \CPP 3.4g
    \item Solaris 7/x86 using g++ 2.95.2
    \item IRIX 6.5 using CC 7.3.1.1m 
	\end{itemize}

\paragraph{External software and libraries}
The compilation of the visualization component BALLVIEW also requires
the QT library (version 2.x) which is available from
\URL{http://www.troll.no/qt}.

If QT was not installed in one of the standard library paths or the
QT header files were not installed in one of the compiler's default
include directories, use "\option{--with-qt-libs}{\tt{}=DIR}" and
"\option{--with-qt-incl}{\tt{}=DIR}" as
options to configure (see Section \ref{section:building-ball}) to specify the paths
QT was installed to.\index{QT}
Please remember to compile also \file{libqgl.a}, which is one of the QT extensions
(to build \file{libqgl.a}, cd to {\tt\nobreak{\$QTDIR/extensions/opengl/src}} and type {\tt
make}).

QT also requires OpenGL. On platforms that do not provide OpenGL, MESA can
be used (e.g. Linux). Mesa is a 3-D graphics library with an API which is 
very similar to that of OpenGL. It can be obtained from \URL{http://www.mesa3d.org}.
\index{MESA}
If MESA is used, please call configure with the option "\option{--with-mesa}"

\section{Installation}
\label{section:building-ball}

Building BALL is very easy, but please read through this section carefully to
avoid any problems.  If all requirements stated above are met, BALL is built
by issuing the following commands in the directory {\tt BALL/source}:

\begin{lstlisting}{}
  ./configure
  make
\end{lstlisting}

The following sections give further details on the configuration of the library,
on the library files created, how to test the library, and how to build BALL 
applications.

\subsection{Configuring BALL}
\index{configure!usage}

"{\tt configure}" tries to gather as much information on your system as possible and 
then creates the necessary configuration files (\file{config.h},
\file{config.mak}, \file{common.mak}, and \file{Makefile}).
The configuration of BALL may be adapted to your needs and to your system
configuration from the command line by adding one or more of the options from
Table \ref{table:options}.
An overview of these options can also be obtained by executing "{\tt configure
--help}"

\begin{table}
\begin{center}
\begin{tabular}{lp{7cm}}\hline
  \option{--x-includes}{\tt{}=DIR}&        X include files are in DIR\\\vspace{3mm}

  \option{--x-libraries}{\tt{}=DIR}&       X library files are in DIR\\\vspace{3mm}

  \option{--enable-optimization}& optimize the library for speed. Remove debug info\\\vspace{3mm}

  \option{--disable-debuginfo}& remove -g from the compiler flags (omit debug information)\\\vspace{3mm}

  \option{--disable-BALLVIEW}& disable the compilation of BALLVIEW, the visualization
												component\\\vspace{3mm}

  \option{--enable-64}&             build 64 bit objects (currently only supported for IRIX MipsPro 
                          compiler and SUNPro compiler)\\\vspace{3mm}

  \option{--with-compiler}{\tt{}=CXX}&     use CXX to compile BALL\\\vspace{3mm}

  \option{--with-cxxflags}{\tt{}=FLAGS}&   add \CPP compiler FLAGS to compile BALL\\\vspace{3mm}

  \option{--with-ldflags}{\tt{}=FLAGS}&    add FLAGS to the linker flags used to link the library and
        									                 applications\\\vspace{3mm}

  \option{--with-qt-incl}{\tt{}=DIR}&      QT header files are in DIR\\\vspace{3mm}

  \option{--with-qt-libs}{\tt{}=DIR}&      QT libraries are in DIR\\\vspace{3mm}

  \option{--with-opengl-incl}{\tt{}=DIR}&  OpenGL/Mesa header files are in DIR/GL\\\vspace{3mm}

  \option{--with-opengl-libs}{\tt{}=DIR}&  	OpenGL/Mesa libraries are in DIR/GL\\\vspace{3mm}

  \option{--with-mesa}&    				         	use MESA instead of OpenGL
								                  	       	If this option is specified, configure looks for libMesaGLU/libMesaGL.
									                         	If no libraries with these names are found, configure looks for
                  									       	libGLU and libGL.
									                         	This switch also add X11 libraries when trying to link agains the Mesa
									                         	libraries.\\\vspace{3mm}

  \option{--without-libxnet}&							 	use \Index{libsocket}/\Index{libnsl} for linking
										  										 	rather than \Index{libxnet}
							  	                         	this option is useful if some of your libraries (e.g. X11 libs)
									                         	were linked against libsocket/libnsl
																					 	\\\vspace{3mm}

  \option{--help}&                  display help information\\\hline
\end{tabular}
\end{center}
\caption{options for {\tt configure}}
\label{table:options}
\end{table}

For example, to compile BALL without the visualization component BALLVIEW,
specify 
\begin{lstlisting}{}
  configure --without-BALLVIEW
\end{lstlisting}

To include BALLVIEW using the QT installation in /opt/qt/lib and /opt/qt/include, specify

\begin{lstlisting}{}
  configure --with-qt-libs=/opt/qt/lib 
		--with-qt-incl=/opt/qt/include
\end{lstlisting}

If Mesa should be used (when compiling under Linux), the correct options might look
like this:

\begin{lstlisting}{}	
  configure --with-qt-libs=/opt/qt/lib 
		--with-qt-incl=/opt/qt/include --without-opengl
    --with-opengl-libs=/opt/mesa/lib 
		--with-opengl-incl=/opt/mesa/include}
\end{lstlisting}

\subsection{Building the Libraries}

After the successful termination of configure, "make" builds the shared libraries.
There exist three different libraries:
\begin{center}
	\begin{tabular}{ll}
  	\file{libBALL.so}&     the main BALL library\\
  	\file{libVIEW.so}&     the visualization component BALLVIEW\\
	  \file{libMOLVIEW.so}&  the molecule--related stuff of BALLVIEW\\
	\end{tabular}
\end{center}

The latter two libraries are not built if "\option{--without-BALLVIEW}" is specified or configure
cannot find X libraries, OpenGL libraries, or QT libraries (and the respective headers).

It is also possible (although not recommended) to build the corresponding static libraries
\file{libBALL.a}, \file{libVIEW.a}, and \file{libMOLVIEW.a} using "{\tt make
staticlibs}".

\subsection{Installing the Libraries}

After compiling the libraries, they are installed in {\tt BALL/lib/\${BINFMT}/}
when calling "{\tt make install}" where {\tt \${BINFMT}} is the binary format
as determined by {\tt configure}.  Currently, the only way to install the
libraries somewhere else is by moving them by hand to the desired destination.
Wherever you install the shared libraries, please make sure to include their
location in the \Index{{\tt LD\_LIBRARY\_PATH}} environment variable.

If you are using \Index{csh}, \Index{tcsh}, or similar shells, use the command
\begin{lstlisting}{}
   setenv LD_LIBRARY_PATH DIR
\end{lstlisting}

\noindent to set the library path. If you are using \Index{sh}, \Index{bash},
or related shells, try

\begin{lstlisting}{}   
   LD_LIBRARY_PATH=DIR
   export LD_LIBRARY_PATH
\end{lstlisting}

If you installed the shared libraries in a directory that the dynamic linker
\Index{ld} searches by default, it is not necessary to set {\tt
LD\_LIBRARY\_PATH}.
