\section{System Requirements}

\subsection{Compiler}
\index{compiler}
  BALL requires a (more or less) ANSI compliant \CPP compiler.
  It has been successfully built and tested on the following platforms:
	\begin{itemize}	
   	\item Linux/x86 2.x using g{\tt ++} 3.2.1
   	\item Linux/x86 2.x using g{\tt ++} 2.95.3
%   	\item Linux/x86 2.x using Kuck \& Associate (KAI) \CPP 4.0
%   	\item Linux/x86 2.x using Intel Compiler 5.0
   	\item Solaris/SPARC 8 using g{\tt ++} 3.2.1
   	\item Solaris/SPARC 8 using g{\tt ++} 2.95.3
%   	\item Solaris/SPARC 8 using Kuck \& Associate (KAI) \CPP 4.0
   	\item Solaris/SPARC 8 using Sun Forte Developer 7 C{\tt ++} 5.4 2002/03/09
   	\item IRIX 6.5 using CC 7.3.1.1m (32 and 64 bit)
%   	\item Compaq Tru64 Unix V4.0f using Compaq \CPP 6.3
		\item Microsoft Windows XP using Microsoft Visual Studio .NET (MSVC 7.0)
 	\end{itemize}

\subsection{External software and libraries}
BALL needs {\tt flex} and {\tt bison} for automatically generating parsers
for various purposes. These utilities are standard software and should be
installed on every contemporary UN*X machine. The newest versions are
downloadable from \URL{http://www.gnu.org/software/}.

The usage of GNU {\tt make} is recommended, although BALL will also build with
other versions of {\tt make}. It is available from \URL{ftp://ftp.gnu.org/gnu/}
and easy to install.

The compilation of the visualization component VIEW also requires
the QT library (version 3.0.6) which is available from
\URL{http://www.troll.no/qt}/.

If QT was not installed in one of the standard library paths or the
QT header files were not installed in one of the compiler's default
include directories, use \mbox{"\option{--with-qt-libs}{\tt{}=DIR}"} and
\mbox{"\option{--with-qt-incl}{\tt{}=DIR}"} as options to configure (see
Section \ref{section:building-ball}) to specify the paths QT was installed
to.\index{QT}
Please remember to compile also \file{libqgl.so}, which is one of the QT
extensions (to build \file{libqgl.a}, cd to 
{\tt\nobreak{\$QTDIR/extensions/opengl/src}} and type {\tt make}).

QT also requires OpenGL. On platforms that do not provide OpenGL, MESA can
be used (e.g. Linux). Mesa is a 3-D graphics library with an API which is 
very similar to that of OpenGL. It can be obtained from 
\URL{http://www.mesa3d.org}.
\index{MESA}

To use the Python extensions of BALL (Python is an object oriented
scripting language), you will also need Python 2.1 installed
(\URL{http://www.python.org}) and a special version of SIP 3.0 distributed
through our website (\URL{http://www.bioinf.uni-sb.de/OK/BALL/}). SIP is a tool
for generating Python bindings for C++ class libraries.

Additionally you might need the FFTW package for fast Fourier
transformations (Verson 2.1.3), available from \URL{http://www.fftw.org/}.

Please make sure that the required external \CPP libraries (i.e. QT and SIP)
have been compiled with the same compiler (and compiler version!) as the BALL
libraries. Otherwise you will most likely see a plethora of strange error
messages, either while linking applications or at runtime.

\section{Installation}
\label{section:building-ball}

Building BALL is very easy, but please read through this section carefully to
avoid any problems.  If all requirements stated above are met, BALL is built
by issuing the following commands in the directory {\tt BALL/source}:

\begin{lstlisting}{}
  ./configure
  make
\end{lstlisting}

The following sections give further details on the configuration of the library,
on the library files created, how to test the library, and how to build BALL 
applications.

\subsection{Configuring BALL}
\index{configure!usage}

"{\tt configure}" tries to gather as much information on your system as possible and 
then creates the necessary configuration files (\file{config.h},
\file{config.mak}, \file{common.mak}, and \file{Makefile}).
The configuration of BALL may be adapted to your needs and to your system
configuration from the command line by adding one or more of the options from
Table \ref{table:options}.
An overview of these options can also be obtained by executing "{\tt configure
--help}"

% \begin{center}
\begin{longtable}{lp{7cm}}\hline
  \option{--x-includes}{\tt{}=DIR}&        X include files are in DIR\\\vspace{3mm}

  \option{--x-libraries}{\tt{}=DIR}&       X library files are in DIR\\\vspace{3mm}

  \option{--enable-optimization}&          optimize the library for speed, remove debug info\\\vspace{3mm}

  \option{--enable-debuginfo}&             create debug information\\\vspace{3mm}

  \option{--disable-VIEW}&                 disable the compilation of the visualization
                                           classes\\\vspace{3mm}

  \option{--enable-64}&                    build 64 bit objects (if allowed
                                           by the compiler)\\\vspace{3mm}

  \option{--with-compiler}{\tt{}=CXX}& use CXX to compile BALL\\\vspace{3mm}

  \option{--with-cxxflags}{\tt{}=FLAGS}&   add FLAGS to the \CPP compiler flags
                                           (commas are converted to blanks)
                                           \\\vspace{3mm}

  \option{--with-ldflags}{\tt{}=FLAGS}&    add FLAGS to the linker flags
                                           (commas are converted to blanks)
                                           \\\vspace{3mm}

  \option{--with-arflags}{\tt{}=FLAGS}&    add FLAGS to the flags for the
                                           creation of the static libraries
                                           \\\vspace{3mm}

  \option{--with-dynarflags}{\tt{}=FLAGS}& add FLAGS to the flags for the
                                           creation of the shared libraries
                                           \\\vspace{3mm}

  \option{--with-qt-incl}{\tt{}=DIR}&      QT header files are in DIR\\
                                           \vspace{3mm}

  \option{--with-qt-libs}{\tt{}=DIR}&      QT libraries are in DIR\\\vspace{3mm}
	\option{--with-moc}{\tt{}=MOC}& 					The absolute path to the QT meta object
																						compiler (moc, typically found in
																						{\tt\$QTDIR/bin/moc})\\\vspace{3mm}

	\option{--with-threadsafe-qt}& 						Set this flag if you the multi-threaded
																						version of Mesa and you compiled
																						the thread-safe version of QT\\\vspace{3mm}
			
  \option{--with-opengl-incl}{\tt{}=DIR}&  OpenGL/Mesa header files are in DIR/GL\\\vspace{3mm}

  \option{--with-opengl-libs}{\tt{}=DIR}&  OpenGL/Mesa libraries are in DIR/GL\\\vspace{3mm}

  \option{--with-mesa}&                    use MESA instead of OpenGL\\
                                           \vspace{3mm}

  \option{--without-libxnet}&              use \Index{libsocket}/\Index{libnsl}
                                           for linking rather than 
                                           \Index{libxnet} (under Solaris)
                                           \\\vspace{3mm}

  \option{--with-python=EXE}& 							Enable Python support using the
																						Python executable in EXE.
Typically, from the executable {\tt configure} can figure out where the
headers and the library are hidden, so that the following two options are
usually not required.\\\vspace{3mm}
  
  \option{--with-python-incl=DIR}&         Python includes (Python.h) is in
                                           DIR\\\vspace{3mm}

  \option{--with-python-libs=DIR}&         Python library (libpython*.a) is
                                           in DIR\\\vspace{3mm}

  \option{--with-python-ldopts=X}&         Use additional options X when
                                           linking with the Python library
                                           \\\vspace{3mm}

  \option{--with-qt-mt}&                   use a threadsafe version of the
                                           QT lib (libqt-mt)
                                           \\\vspace{3mm}

  \option{--with-sip-lib=DIR}&             the SIP library resides in DIR
                                           \\\vspace{3mm}

  \option{--with-sip-incl=DIR}&            the SIP header file resides in DIR
                                           \\\vspace{3mm}

  \option{--with-sip=DIR}&                 the SIP executable resides in DIR
                                           \\\vspace{3mm}

  \option{--without-xdr}&                  no RPC/XDR headers available - do
                                           not build portable binary
                                           persistence support
                                           \\\vspace{3mm}

  \option{--help}&                         display help information\\\hline
\caption{options for {\tt configure}}
\label{table:options}
\end{longtable}
%\end{center}
%\caption{options for {\tt configure}}
%\label{table:options}

For example, to compile BALL without the visualization component,
specify 
\begin{lstlisting}{}
  configure --disable-VIEW
\end{lstlisting}

To compile the visualization classes using the QT installation in /opt/qt/lib and /opt/qt/include, specify

\begin{lstlisting}{}
  configure --with-qt-libs=/opt/qt/lib 
		--with-qt-incl=/opt/qt/include
\end{lstlisting}

If Mesa should be used (when compiling under Linux), the correct options might look
like this:

\begin{lstlisting}{}	
  configure --with-qt-libs=/opt/qt/lib 
		--with-qt-incl=/opt/qt/include
    --with-opengl-libs=/opt/mesa/lib 
		--with-opengl-incl=/opt/mesa/include
\end{lstlisting}

\subsection{Building the Libraries}

After the successful termination of configure, issuing "make" will build the
shared libraries. Two different libraries will be built:
\begin{center}
	\begin{tabular}{ll}
  	\file{libBALL.so}&     the main BALL library\\
  	\file{libVIEW.so}&     the visualization classes\\
	\end{tabular}
\end{center}

The latter two libraries are not built if "\option{--disable-VIEW}" is specified or configure
cannot find X libraries, OpenGL libraries, or QT libraries (and the respective headers).

It is also possible (although not recommended) to build the corresponding static libraries
\file{libBALL.a} and \file{libVIEW.a} using "{\tt make
staticlibs}". Please note that statically linked binaries are huge.

\subsection{Installing the Libraries}

After compiling the libraries, they are installed in {\tt BALL/lib/\${BINFMT}/}
when calling "{\tt make install}" where {\tt \${BINFMT}} is the binary format
as determined by {\tt configure}.  Currently, the only way to install the
libraries somewhere else is by moving them by hand to the desired destination.
Wherever you install the shared libraries, please make sure to include their
location in the \Index{{\tt LD\_LIBRARY\_PATH}} environment variable.

If you are using \Index{csh}, \Index{tcsh}, or similar shells, use the command
\begin{lstlisting}{}
   setenv LD_LIBRARY_PATH DIR
\end{lstlisting}

\noindent to set the library path. If you are using \Index{sh}, \Index{bash},
or related shells, try

\begin{lstlisting}{}   
   LD_LIBRARY_PATH=DIR
   export LD_LIBRARY_PATH
\end{lstlisting}

If you installed the shared libraries in a directory that the dynamic linker
\Index{ld} searches by default, it is not necessary to set {\tt
LD\_LIBRARY\_PATH}.
