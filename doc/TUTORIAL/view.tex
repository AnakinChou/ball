Visualization not only covers the graphical representation of things (\eg
molecules, measurement data, \dots) but also the graphical user interface
(GUI). Both jobs can be done using {\tt VIEW} functionality. Both areas are
discussed in the next few sections. First we will learn how to create
geometric primitives one can use for graphical representations. After that
building a GUI will be shown by creating a user dialog as example.

Let us turn to the graphical representation. In VIEW there are a number of
predefined geometric primitives already available, \eg {\em Sphere}, {\em
Tube} etc. But sometimes a needed primitive may not be available and
therefore must be programmed anew. So now we will show how this can be
achieved. A new primitive, the cross, will be introduced.

\section{How to create a geometric primitive}
\label{section:view_create_a_geometric_primitive}

In this section we want to create a new geometric primitive called 'Cross'.
We define a cross to be a shape that consists of three lines that merge in one
point. Additionally we require all lines to be axis aligned and meet each
other in the middle.

To accomplish this we need three properties for the geometric object: the
float member radius that describes the half length of each line, the class 
\class{Vertex} for the middle point of the geometric primitive and the class 
\class{ColorExtension} which contains methods for changing the color of the
cross. In addition to these classes we need the main base class for
creating a geometric primitive: The \class{GeometricObject} 
implements the interface each geometric shape must have.

The definition of \class{Cross} looks as follows:
\begin{lstlisting}{}
class Cross: 
	public ColorExtension,
	public Vertex,
	public GeometricObject
{
	public:

		Cross() throw();

		virtual ~Cross() throw();
		
		float getRadius() const throw();

		void setRadius(float new_radius) throw();

	protected:
					
		float radius_;
};
\end{lstlisting}

As this object is derived from all the base classes, we only need to implement
a standard constructor, the destructor and the get- and set- methods for 
the radius.
\footnote{The copy constructor and the copy assignment methods
have been omitted because they are not crucial to the implementation of a
primitive.} All additional functionality is provided by inheritance.

We will now have a closer look at the implementation of the drawing method. 
To be able to draw the new geometric object class, we have to add the new
method renderCross\_ to the classes \class{Renderer} and \class{GLRenderer}.

The method Renderer::render\_ defines which drawing methods are called for
which geometric objects. We add the four new lines at the bottom, 
so it recognizes the new class \class{Cross}.

\begin{lstlisting}{}
void Renderer::render_(const GeometricObject* object)
 throw()
{
  if      (RTTI::isKindOf<Sphere>(*object))         
  { 
		renderSphere_(*(const Sphere*) object);
	}
  else if (RTTI::isKindOf<TwoColoredLine>(*object)) 
	{ 
		renderTwoColoredLine_(*(const TwoColoredLine*) object);
	}
  else if (RTTI::isKindOf<Cross>(*object))          
	{ 
		renderCross_(*(const Cross*) object);
	}
  ...
\end{lstlisting}

The method Renderer::renderCross\_(const Cross\& cross)
will be overloaded by derived Renderer classes, so it only
contains a warning, which will appear if we forget to 
implement it in a derived Renderer:

\begin{lstlisting}{}
virtual void renderCross_(const Cross& /* cross */)
	throw() 
{
	Log.error() << "renderCross_ not implemented in derived Renderer class" << std::endl;
}
\end{lstlisting}


The method GLRenderer::renderCross\_(const Cross\& cross)
does the actual rendering, so we use OpenGL code here:

\begin{lstlisting}{}
void GLRenderer::renderCross_(const Cross& cross)
	throw() 
{
	glPushMatrix();
	
	// if cross is selected, use the selection color,
	// otherwise use its own color. (method from GLRenderer)
	setColor4ub_(cross);  
	
	// move to the position of the cross (method from GLRenderer)
	translateVector3_(sphere.getVertex());
	
	// OpenGL code for rendering the cross.
	glBegin(GL_LINES);
	glVertex3f((GLfloat)(getVertex().x - cross.getRadius()),
						 (GLfloat)(getVertex().y),
						 (GLfloat)(getVertex().z));
	glVertex3f((GLfloat)(getVertex().x + cross.getRadius()),
						 (GLfloat)(getVertex().y),
						 (GLfloat)(getVertex().z));

	glVertex3f((GLfloat)(getVertex().x),
						 (GLfloat)(getVertex().y - cross.getRadius()),
						 (GLfloat)(getVertex().z));
	glVertex3f((GLfloat)(getVertex().x),
						 (GLfloat)(getVertex().y + cross.getRadius()),
						 (GLfloat)(getVertex().z));

	glVertex3f((GLfloat)(getVertex().x),
						 (GLfloat)(getVertex().y),
						 (GLfloat)(getVertex().z - cross.getRadius()));
	glVertex3f((GLfloat)(getVertex().x),
						 (GLfloat)(getVertex().y),
						 (GLfloat)(getVertex().z + cross.getRadius()));
	glEnd();

	glPopMatrix();
}
\end{lstlisting}


\section{Construction of a dialog}
\label{section:view_construction_of_a_dialog}

A crucial step in implementing an application is the programming of dialogs or
widgets, their interaction with other widgets and the integration into the
main application. We will now discuss the creation of a dialog that will have
all the above properties. A widget can be created in a similar manner, by
exchanging the \class{QDialog} base class with the \class{QWidget} class. That
should do the trick.

The construction of a dialog/widget is somewhat complicated because of the
many actions it can perform. A dialog can have menu entries in the main
application (derived from the \class{MainControl}) and can react to messages
sent from other dialogs/widgets (see \class{Message} and
\class{ModularWidget}). We want to create a dialog which has both of the
above properties. As an example we choose the \class{LabelDialog}, which already
exists in the VIEW library. This dialog can add Labels to the visualisation of
molecules.
We created the GUI of the dialog with the program "QT Designer" which is part
of every QT-package. This has the advantage, that the creation of the GUI takes
less time and the layout can be changed with much less effort. 
The result of the "designer" program is a .ui file, which the program "uic"
from the QT-library transforms to source files. We derive from the class
LabelDialogData, that is defined in these sources.

The menu entry "Add Label" for our dialog will toggle the visibility of the dialog itself.
The color for the labels will be written to
and read from the \class{INIFile}, which the main application keeps. The
\class{Message} that should trigger some effect will be the message 
\class{ControlSelectionMessage} that is sent by the class \class{MolecularControl} 
whenever the selection changes. To accomlish all these things we have to 
override certain methods.

First we have a look to the header file of our dialog (includes are omitted).
See \class{ModularWidget} for information about the various methods that must
be overridden.

\begin{lstlisting}{}

class LabelDialog : 
	public LabelDialogData,
	public ModularWidget
{
	Q_OBJECT
	
	// needed for getInstance() and for Python Interface
	BALL_EMBEDDABLE(LabelDialog,ModularWidget)
		
	public:
	
	LabelDialog(QWidget *parent = NULL, const char *name = NULL )
		throw();

	virtual ~LabelDialog()
		throw();
					
	virtual void onNotify(Message *message)
		throw();
					
	virtual void fetchPreferences(INIFile &inifile)
		throw();
	
	virtual void writePreferences(INIFile &inifile)
		throw();
		
	virtual void initializeWidget(MainControl& main_control)
		throw();
	
	virtual void finalizeWidget(MainControl& main_control)
		throw();
				
	public slots:

	void show();

	protected slots:
					
	virtual void accept();
	virtual void editColor();

private:
	
	int id_;
	
	ColorRGBA custom_color_;
	List<Composite*> selection_;
};
\end{lstlisting}

Now we discuss the implementation. Before we handle the initialization,
update and the removal of the menu entries let's have a look to the
implementation of the constructor. First we set the caption of this dialog
and register it as a modular widget. This action is very important because it
creates the internal mechanism that connects all widgets with each other. If
you create your own dialogs/widgets which have to react to messages always make 
sure this method is called.

\begin{lstlisting}{}
LabelDialog::LabelDialog(QWidget* parent, const char* name)
	throw()
	:	LabelDialogData( parent, name ),
		ModularWidget(name)
{
	setCaption("Add Label");

	// register the widget with the MainControl
	ModularWidget::registerWidget(this);

	hide();
}
\end{lstlisting}

The next method creates the menu entry for our dialog and connects it to a slot
that will open our dialog. First we create the menu {\em DISPLAY} if it is not
already created and set the flag {\em checkable} to assure that all entries
can have a check flag.
Then we create the menu entry {\em Add Label} in the menu {\em DISPLAY} and
connect it with the slot LabelDialog::show(), which is provided by the class QWidget. 
The id we get from this method is stored in the variable id\_. We need this
id to check the menu entry if this dialog is open.

\begin{lstlisting}{}
void LabelDialog::initializeWidget(MainControl& main_control)
	throw()
{
	main_control.initPopupMenu(MainControl::DISPLAY)->setCheckable(true);

	id_ = main_control.insertMenuEntry(MainControl::DISPLAY, "Add &Label", this,
																		 SLOT(show()), CTRL+Key_L, -1,
																		 "Add a label for selected molecular objects");   
}

\end{lstlisting}

Finally, if the dialog is to be destroyed the menu entry will be removed from
the main application (derived from {\em MainControl}).
\begin{lstlisting}{}
void LabelDialog::finalizeWidget(MainControl& main_control)
	throw()
{
	main_control.removeMenuEntry(MainControl::DISPLAY, "Add &Label", this,
															 SLOT(show()), CTRL+Key_L);   
}
\end{lstlisting}

In the above methods it as also possible to do other initialization or menu handling stuff.
It is also possible to add, update and remove more menu entries (for each menu entry
there must be a slot to be connected to).


The next methods we will discuss are the methods responsible for reading and
writing the contents of this widget into the \class{INIFile}. First we look at the
method responsible form reading the preferences.
If the \class{INIFile} has the section {\tt WINDOWS} and the key {\tt Label::customcolor} then
we read the contents of this key and convert it to a color which we store in
\member{custom\_color\_}{} (see \class{ColorRGBA}). 
Additionally we assign the color to the label to be displayed in the widget.

\begin{lstlisting}{}
void LabelDialog::fetchPreferences(INIFile& inifile)
	throw()
{
	if (inifile.hasEntry("WINDOWS", "Label::customcolor"))
	{
		custom_color_.set(inifile.getValue("WINDOWS", "Label::customcolor"));
		color_sample_->setBackgroundColor(custom_color_.getQColor());
	}
	else
	{
		custom_color_.set(ColorRGBA(1.,1.,0.,1.));
		color_sample_->setBackgroundColor(custom_color_.getQColor());
	}			
}
\end{lstlisting}

To write the preferences to the inifile we override this method.

\begin{lstlisting}{}
void LabelDialog::writePreferences(INIFile& inifile)
	throw()
{
	inifile.insertValue("WINDOWS", "Label::customcolor", custom_color_);
}
\end{lstlisting}

{\em Note:} It is important that the section {\tt WINDOWS} is
already inserted into the inifile, else the inifile will not store the color.
See \class{INIFile} for further information.

The last method of the preferences tab widget is the method editColor  that
is called if the edit button is pressed.
First we call the dialog \class{QColorDialog} where we can change the color.
If this dialog was exited by pressing the ok button we transfer the color into
the color label. Then we store it in the variable custom\_color\_.

\begin{lstlisting}{}
void LabelDialog::editColor()
{
	color_sample_->setBackgroundColor(
			QColorDialog::getColor(color_sample_->backgroundColor()));
	QColor qcolor = color_sample_->backgroundColor();

	custom_color_.set(qcolor);
	update();
}
\end{lstlisting}

The slot accept is connected to the apply button and will add the label
to the visualisation of the selected molecular objects:
\begin{lstlisting}{}
void LabelDialog::accept()
{
	// no selection present => return
	if (selection_.empty()) return;

	// calculate the geometric center for the selection
	GeometricCenterProcessor center_processor;
	
	// center to which the label will be attached
	Vector3 center;

	// process all objects in the selection list
	List<Composite*>::Iterator list_it = selection_.begin();
	for (; list_it != selection_.end(); ++list_it)
	{
		(*list_it)->apply(center_processor);
		center += center_processor.getCenter();
	}

	center /= selection_.size();

	// create a new Label 
	Label* label = new Label;
	label->setText(label_edit_->text().ascii());
	label->setColor(custom_color_);
	label->setVertex(center);

	// create a new Representation
	Representation* rep = getMainControl()->
				getPrimitiveManager().createRepresentation();
	rep->insert(*label);
	rep->setProperty(Representation::PROPERTY__ALWAYS_FRONT);
	rep->setModelType(MODEL_LABEL);

	// create and send a RepresentationMessage to inform the Scene and
	// update the GeometricControl
	RepresentationMessage* rm = new RepresentationMessage;
	rm->setRepresentation(*rep);
	rm->setType(RepresentationMessage::ADD);
	notify_(rm);
	
	setStatusbarText("Label added.");
}
\end{lstlisting}

The only thing that still must be implemented is the mechanism that handles
any messages received from our dialog. This is done in the method
onNotify. 
We perform a runtime type identification of the received message and react only if the message is of
type \class{SelectionMessage}.
We enable the apply button in our dialog if the selection list contains any
elements. Otherwise we disable it.

\begin{lstlisting}{}
void LabelDialog::onNotify(Message *message)
	throw()
{
	// selection => store last selection for later processing
	if (RTTI::isKindOf<ControlSelectionMessage>(*message))
	{
		ControlSelectionMessage* selection = 
			RTTI::castTo<ControlSelectionMessage>(*message);
		selection_ = selection->getSelection();
	}

	// disable apply button, if selection is empty
	apply_button_->setEnabled(!selection_.empty());
	menuBar()->setItemEnabled(id_, !selection_.empty());
}
\end{lstlisting}


Now we are finished with the construction of this dialog. It can be added to
the main application by creating it with a pointer to the application ({\em
MainControl}) as parent.

