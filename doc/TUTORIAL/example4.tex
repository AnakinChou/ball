\section{The BALL File Class}

All classes handling file I/O in BALL are derived from a common
base class, \class{File}. This base class provides a lot of
functionality that applies to all derived classes as well.
One of the most useful features is on-the-fly file transformation.
For example, we can open a file (\eg a PDB file using the \class{PDBFile} 
class) that is not stored locally on a disc, but in the internet:

\begin{lstlisting}{}
  PDBFile
  infile("ftp://ftp.mpi-sb.mpg.de/pub/outgoing/BALL/pdb4pti.ent");
  System S;
  infile >> S;
  infile.close();
\end{lstlisting}

\noindent
This command retrieves the file {\tt pdb4pti.ent} from its location at the MPI
site using the FTP protocol and reads the contents of that file into a
\class{System}. The retrieval and the expansion of the URL into something
meaningful is performed by the classes \class{TransformationManager} and
\class{TCPTransfer}. Any filename is first handed to the static instance of
\class{TransformationManager} which \class{File} possesses, which then applies
all matching rules from a predefined rule set to that file name. The resulting
expanded file names are then checked for special prefixes by \class{File}:
\begin{itemize}
  \item a file name starting with \ttindex{exec:} will execute the command
        after the colon, redirect the output of the command to a temporary
        file, which is then opened and returned,

  \item a file name starting with \ttindex{http:} or \ttindex{ftp:} will
        initiate an HTTP or FTP tranfer from the given URL to a temporary file,
        which is opened and returned.
\end{itemize}

\noindent 
The {\tt exec:} prefix can be used to filter existing files through a command, 
e.g. the compressed (GZIPped) file {\tt test.txt.gz} can be achieved through
the filename {\tt exec:gunzip -c test.txt.gz}, assuming that the gunzip
executable is in your path. You can also automate this process using the
\class{TransformationManager}. By simply defining a rule for all files ending
in {\tt .gz}, gunzip will be called automatically:

\begin{lstlisting}{}
	File::registerTransformation
		(".*\.gz", "exec:/usr/local/bin/gunzip -c %s");
\end{lstlisting}

\noindent
Similarly, if you store a local copy of the PDB in {\tt /local/PDB}, you might
want to define a short-hand for the path to the PDB through the following
rule:

\begin{lstlisting}{}
	File::registerTransformation
		("PDB:.*", "/local/PDB/structures/all/pdb/pdb%b.ent");
\end{lstlisting}

\noindent
This rule would then expand the name {\tt PDB://4pti} to the local copy
in {\tt /local/PDB/structures/all/pdb/pdb4pti.ent}. In the above rule, 
{\tt 4pti} corresponds  to the basename of the file, \ie the name without path 
and without the file type extension (everything after the last dot).
For details on the formulation of rules, please refer to the section
covering \class{TransformationManager} in the BALL Reference Manual.

Some restrictions apply when using transformed file names. First, they may
be used for {\em reading} files only -- there is no distinct rule set for
writing files. Second, the transformation manager will always apply the first 
rule it finds. Therefore chaining of rules is not possible and the user
is responsible for avoiding ambiguities between multiply defined rules.
