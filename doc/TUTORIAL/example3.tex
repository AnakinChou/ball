\section{A simple AMBER calculation}

Having introduced the basic of handling proteins in the last
chapter, we now turn towards real-life examples: getting a protein
from the PDB and performing an AMBER calculation with it.
Again, we will be using BPTI. Instead of reading BPTI from a 
file, as in the last example, we will use the TCP transfer
capabilities of BALL to retrieve the file directly form the 
PDB (if your machine is connected to the internet -- you might
want to read it from a file as in the last example otherwise).
All classes handling file I/O in BALL are derived from a common
base class, \class{File}. This base class provides a lot of
functionality that applies to all derived classes as well.
On eof the most usefult features is on-the-fly file transformation.
Iin our case, we can open a file (e.g. a PDB file using the \class{PDBFile} 
class) that is not stored locally on a disc, but in the internet:

\begin{lstlisting}{}
	PDBFile
	infile("ftp://ftp.rcsb.org/pub/pdb/data/structures/all/pdb/pdb4pti.ent");
	System S;
	infile >> S;
	infile.close();
\end{lstlisting}

\noindent
This command retrieves the file {\tt pdb4pti.ent} from its location at the RCSB
site using the FTP protocol and reads the contents of that file into a
\class{System}. The retrieval and the expansion of the URL into something
meaningful is performed by the classes \class{TransformationManager} and
\class{TCPTransfer}. Any filename is first handed to the static instance of
\class{TransformationManager} \class{File} possesses and all 

\begin{lstlisting}{}
	File::registerTransformation(".*\.gz", "exec:/usr/local/bin/gunzip -c %s");
\end{lstlisting}

\noindent
