\section{Overview}

\Index{Python} is an object-oriented scripting language~\cite{Python} that is 
well-suited both as a language for embedding scripts into BALL applications and
as a rapid prototyping language using the underlying BALL objects.
BALL provides Python bindings for most of its classes in order to allow 
\Index{Rapid Methodology Development} (RMD). Some of the functionality BALL 
provides (\eg template classes, iterators) is unavailable due to the 
fundamental differences between the two languages. However, the majority of 
the classes is available and workarounds exist for some of the template- and 
iterator-related problems.

Since release 1.2 of BALL, the Python support is enabled by default. The 
remainder of this section assumes that you are somewhat familiar with the most
important language concepts of Python.

BALL relies on SIP~\cite{SIP} \index{SIP} to translate its class
headers semi-automatically into Python wrapper classes. For each \CPP class
SIP creates a subclass defining the Python interface and a Python class
using that \CPP interface class. The Python class has the same name as the
\CPP class, so porting code from \CPP to Python (and vice versa) gets trivial.
The \CPP code 

\begin{lstlisting}{}
  System S;
  HINFile f("test.hin");
  f >> S;
\end{lstlisting}

\noindent
translates to the Python code

\begin{lstlisting}{}
  S = System()
  f = HINFile("test.hin")
  f >> S;
\end{lstlisting}

\noindent
In this example, the main difference is how \CPP and Python handle
constructors. Another important difference concerns iterators. The STL-like
kernel iterators of BALL map to a set of functions (called extractors). An
extractor traverses the whole container and creates a Python sequence object
from it. Instead of having an \class{AtomIterator} iterating over all atoms of
a residue

\begin{lstlisting}{}
  AtomIterator ai = residue.beginAtom();
  for (; +ai; ++ai)
  {
    std::cout << ai->getName() << std::endl;
  }
\end{lstlisting}

\noindent
an \function{atoms} extractor is used to create a sequence object containing
all objects of the residue in Python:

\begin{lstlisting}{}
  for atom in atoms(residue):
    print atom.getName()
\end{lstlisting}

For the template problem, we pre-instantiated some of the 
commonly used instances, \eg \class{Unary\-Processor\-<Atom>} maps to the 
\class{AtomProcessor} class and classes derived from it in Python.

The BALLView application contains an interactive interpreter window
if BALL was compiled with Python support. You can even access the
data structures of the viewer from there. Assuming that you are
currently displaying a structure in the viewer, you can retrieve a
reference to the first system displayed through the somewhat cryptic
command 

\begin{lstlisting}{}
  system = MainControl::getInstance(0)\\
  .getCompositeManager().getComposites()[0].
\end{lstlisting}

\noindent
Since this is not very convenient, we added a Python startup script that is 
always executed when \mbox{BALLView} starts up. It can be found under 
\file{BALL/data/startup.py}. By using one of the methods defined in this file,
it is possible to obtain the first \class{System} by simply calling

\begin{lstlisting}{}
  system = getSystem(0).
\end{lstlisting}

\noindent
This is very useful for extracting properties of loaded molecules and other 
datasets you are currently displaying, but not recommended if you start 
modifying internals of the viewer. You should also not try to destroy those 
objects in the viewer, or you will be rewarded with a core dump. Currently 
there is no further documentation of the Python support available.
		

\section{Installation}

The \texttt{BALLCore} and \texttt{VIEW} Python modules are already included
in all source and  binary distributions and installed automatically.
Alternatively, the modules can also be built manually using the
\texttt{BALLCoremodule} and \texttt{VIEWmodule} targets:
\begin{lstlisting}{language=sh}
make BALLCoremodule VIEWmodule
\end{lstlisting}
On Windows, the modules can be built using the \texttt{BALLPython.sln}
solution file.

Before the modules can finally be used, two environment variables need to be set:
\begin{lstlisting}{language=sh}
# Linux/Mac
export PYTHONPATH=<PATH to BALL libs>:<PATH to sip libs>:${PYTHONPATH}
export LD_LIBRARY_PATH=<PATH to BALL libs>:<PATH to QT libs>:\
<PATH to sip libs>:${LD_LIBRARY_PATH}

# Windows
set PYTHONPATH=<PATH to BALL libs>;<PATH to sip libs>;%PYTHONPATH%
set PATH=<PATH to BALL libs>;<PATH to QT libs>;<PATH to sip libs>;%PATH%
\end{lstlisting}

