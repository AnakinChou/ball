\section{Overview}
Python is an object-oriented scripting language\cite{Python} that is well
suited both as a language for embedding scripts into BALL application and as a
rapid prototyping language using the underlying BALL objects.
BALL provides Python bindings for most of its classes in order to allow Rapid
Methodology Development (RMD). Some of the functionality BALL provides (\eg
template classes, iterators) are unavailable due to the fundamental differences 
between the two languages. However, the majority of the classes is available and
workarounds exist for some of the template- and iterator-related problems.

The current Python support is not yet fully tested and debugged, therefore it
is disabled by default. If you can live with
this and think it is still a useful feature worth testing out, please read on,
otherwise pretend it doesn't exist. The remainder of this section assumes that
you are somewhat familiar with the most important language concepts of Python.

BALL relies on SIP \cite{SIP} (version 4.3.1 or above) to translate its class
headers semi-automatically into python wrapper classes. For each \CPP class
SIP creates a subclass defining the Python interface and a Python class
using that \CPP interface class. The Python class has the same name as the
\CPP class, so porting code from \CPP to Python (and vice versa) gets trivial.
The \CPP code 

\begin{lstlisting}{}
	System S;
	HINFile f("test.hin");
	f >> S;
\end{lstlisting}

\noindent
translates to the Python code

\begin{lstlisting}{}
	S = System()
	f = HINFile("test.hin")
	f >> S;
\end{lstlisting}

\noindent
In this example, the main difference lies in the way \CPP and Python handle
constructors. Another important difference concerns iterators. The STL-like
kernel iterators of BALL map to a set of functions (called extractors). An
extractor traverses the whole container and creates a Python sequence object
from it. Instead of having an \class{AtomIterator} iterating over all atoms of
a residue

\begin{lstlisting}{}
	AtomIterator ai = residue.beginAtom();
	for (; +ai; ++ai)
	{
		std::cout << ai->getName() << std::endl;
	}
\end{lstlisting}
\noindent you would use the \function{atoms} extractor to create a sequence
object containing all objects of the residue in Python:

\begin{lstlisting}{}
	for atom in atoms(residue):
		print atom.getName()
\end{lstlisting}

\noindent For the template problem, we pre-instantiated some of the 
commonly used instances, \eg \class{UnaryProcessor<Atom>} maps to the 
\class{AtomProcessor} class and classes derived from it in Python.

The BALLView application contains an interactive interpreter window
if BALL was compiled with Python support. You can even access the
data structures of the viewer from there. Assuming that you are
currently displaying a structure in the viewer, you can retrieve a
reference to the first system displayed through the somewhat cryptic
command 
\\
{\tt system = MainControl::getInstance(0)\\
.getCompositeManager().getComposites()[0]}.
Since this is not very convenient, we added a Python startup script,
that will always be run when BALLView starts up. It can be found
under BALL/data/startup.py. By using one of the methods defined in this file,
it is possible to obtain the first System by simply calling \\
{\tt system = getSystem(0)}.
This is pretty useful for extracting properties from the loaded molecules
and other datasets you are
currently displaying, but very deadly if you start modifying internals
of the viewer. You also should not try to destroy those objects in the
viewer, or you will be rewarded with a core dump: there is not yet
internal reference counting on the BALL side, so Python's reference
counting is nearly always wrong.
		

\section{Known bugs}
\begin{itemize}
	\item No documentation yet
	\item No unit tests yet
	\item As mentioned above, problems with reference counting.
	\item Python support requires the visualization support for some rather
				arcane internal reason we won't disclose here.
	\item {\tt configure} does not yet identify SIP and its libraries
				automatically
\end{itemize}

\section{Installation}

Donwload, configure, and install a recent version (4.3.1 or above) of SIP from 
\URL{http://www.riverbankcomputing.co.uk/sip/}. It is important to compile BALL 
with the same \CPP compiler you compiled SIP and QT with. You can then enable the Python support of BALL
by specifying the option \mbox{"\option{--with-python=<path>}"}, where
{\tt path} should point to the executable of an installed Python (version 2.3
preferred). You will also have to specify the path to the SIP executable, its
headers, and its library ({\tt libsip.a|so}). Adding these four options might
look something like this on your system:

\begin{lstlisting}{}
	./configure \
		--with-python=/opt/bin/python2.3\
		--with-sip=/opt/bin/sip\
		--with-sip-incl=/opt/include/sip\
		--with-sip-lib=/opt/lib
\end{lstlisting}

\noindent
After configuring and building BALL, you should then change to {\tt
BALL/source/PYTHON/EXTENSIONS}. Here, you will have to run

\begin{lstlisting}{}
	make sip
	make depend
	make
	make install
\end{lstlisting}

\noindent
in order to build the Python bindings, which are then installed to {\tt
BALL/lib/<platform>}. 
You can then execute the {\tt pyball} script in {\tt
BALL/source/PYTHON/EXTENSIONS} to get an interactive Python shell with the
BALL bindings or just start the Python interpreter and import the BALL
bindings through

\begin{lstlisting}{}
	from BALL import *
\end{lstlisting}

\noindent
Note, that you have to add the directory where the bindings are installed to
Python's module search path ({\tt sys.path}).
