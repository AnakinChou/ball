% -*- Mode: C++; tab-width: 2; -*-
% vi: set ts=2:
%
% $Id: coding_style.tex,v 1.5 2004/02/23 11:09:06 oliver Exp $
%
                                                                                                                                                             
\documentclass[a4paper,10pt]{article}
\usepackage{a4wide}
\usepackage{times}

\title{BALL coding conventions}
\author{Oliver Kohlbacher}

\date{$ $Revision: 1.5 $ $}

\newtheorem{example}{Example}

\begin{document}
\maketitle

\section{General}

\subsection{Formatting}

{\bf All BALL files use a tab width of two.} Use the command {\tt set tabstop=2} in
{\tt vi} or {\tt set-variable tab-width 2} if you are using {\tt emacs}. The
standard file header should automatically set these values for {\tt vi} or
{\tt emacs} (see below).

\noindent
{\bf Matching pairs of opening and closing curly braces should be set to the same
column.}

\begin{verbatim}
while (continue == true)
{
  for (int i = 0; i < 10; i++)
  {
    ...
  }

 	...

  if (x < 7)
  {
    ....
  }
}
\end{verbatim}
The main reason for this rule is to avoid constructions like:

\begin{verbatim}
if (isValid(a))
  return 0;
\end{verbatim}

which might be changed to something like

\begin{verbatim}
if (isValid(a))
  error = 0;
  return 0;
\end{verbatim}

The resulting errors are hard to find. There are two ways to avoid these
problems: (a) always use braces around a block (b) write everyting in a single
line. We recommend method (a).
However, this is mainly a question of personal style, so no explicit checking
is performed to enforce this rule.



\section{Class requirements}

Each Ball class has to provide the following minimal interface:
\begin{verbatim}
	/** Class documentation... 
	*/
  class Test
  {
    public:

    // clone method (implemented automatically with the CREATE macro)
    CREATE(Test)    

    // default ctor
    Test();

    // copy ctor 
    Test(const Test& test);

    // destructor 
    virtual ~Test();
 
    // assignment operator
    Test& operator = (const Test& test);
  };
\end{verbatim}



\section{Naming conventions}


\subsection{Variable names}

{\bf Variable names are all lower case letters.} Distinguished parts of the name are
separated using underscores ``{\tt \_}''. If parts of the name are derived
from common acronyms (e.g. PDB) they should be in upper case.\\
Private or protected member variables of classes are suffixed by an
underscore.\\
No prefixing or suffixing is allowed too identify the variable type - this
leads to completely illegible documentation and overly long variable names.\\
\begin{example}\hspace*{2mm}\\
{\tt String PDB\_atom\_name;} (using an abbreviation)\\
{\tt int counter\_;} (private class member)\\
{\tt bool is\_found;} (ordinary variable)
\end{example}

\subsection{Class names/type names}

{\bf Class names and type names always start with a capital letter.} Different parts
of the name are separated by capital letters at the beginning of the word. No
underscores are allowed in type names and class names, except for the names of
protected types and classes in classes which are suffixed by an underscore.
\begin{example}\hspace*{2mm}\\
{\tt class PDBFile}\\
{\tt class ForwardIteratorTraits\_} (private nested class definition)\\
\end{example}

\subsection{File names}

{\bf Header and source files should be named after the class the contain.}
Each file should contain a single class only, although exceptions are possible
for extreme light-weight classes. The file names start with a lower-case
letter (for historical reasons). Hence, the file containing the {\tt
AtomContainer} class would be named {\tt atomContainer.C}.

\subsection{Function names/method names}

{\bf Function names (including class method names) always start with a lower case
letter.} Parts of the name are separated using capital letters (as are types
and class names). No underscores are allowed in type names and class names,
except for the names of protected types and classes in classes which are
suffixed by an underscore. The argument of {\tt void} functions is omitted in
the declaration and the definition. If function arguments are pointers or
references, the pointer or reference qualifier is appended to the variable
type (without separating blanks). It should not prefix the variable name.\\
The variable names used in the declaration have to be the same as in the
definition. They should be as short as possible, but nevertheless
comprehensible.  If arguments are not used in the implementation of the
function, they have to be commented out (to avoid compiler warnings).
\begin{example}\hspace*{2mm}\\
{\tt int countAtoms()}\\
{\tt bool isBondedTo(const Atom\& atom)}\\
{\tt bool doSomething(int i, String\& /* name */)}
\end{example}

\section{Documentation}

\subsection{Doxygen}
{\bf Each BALL class has to be documented using Doxygen.} The documentation is
inserted in Doxygen format in the header file where the class is defined.
Documentation includes the description of the class, of each method, type
declaration, enum declaration, each constant, and each member variable.
An example for a class documentation is given below. 

\subsection{Usage of groups in the documentation}

In order to obtain a structured and comprehensible documentation methods are
grouped by their function. Each class usually uses one or more out of the
following categories to group its methods:
\begin{itemize}
	\item {\bf Type Definitions:}\\
		This group should contain all {\tt typedef}s (except for private or pretected
		ones) of the class
	\item {\bf Exceptions:}\\
		All class specific exceptions are defined as nested classes in this group
	\item {\bf Enums:}\\
		This group usually contains enums or constants defined via enums such as
		properties
	\item {\bf Constructors and Destructor:}\\
		This group includes all constructors and destructors, as well as related
		methods such as {\tt clear} or {\tt destroy}
	\item {\bf Persistence:}\\
		This group is present in each class derived from {\tt PersistentObject}
		and always contains the two methods {\tt persistentWrite} and {\tt
		persistentRead}
	\item {\bf Converters:}\\
		If an object should be convertible to another class, an explicit
		converter should be defined inside this group.
	\item {\bf Accessors:}\\
		Accessors are all the methods that access or modify attributes of the
		class.
	\item {\bf Predicates:}\\
		Predicates are functions returning a boolean value. the names of
		predicates should start with {\tt is} or {\tt has} if applicable.
		Comparison operators are also predicates and should appear inside
		this group.
	\item {\bf Debugging and Diagnostics:}\\
		This group should contain all methods used for the debugging of the class
		(for example the {\tt dump} methods).
	\item {\bf Attributes:}\\	
		This group should contain all public class attributes. Protected and
		private attributes should be documented in the corresponding groups
		{\bf Private Attributes} or {\bf Protected Attributes}. However, private 
		members do not occur in the documentation by default.
\end{itemize}

These section names are intended as a rule-of-thumb only. There are many cases
where other section headings are more intuitive. However, if any of the above
sections apply, they should be used. The sections should also appear in this
order.

\subsection{Commenting code}

The code for each .C file has to be commented. Each piece of code in BALL has
to contain at least {\bf 5\% of comments}. The use of {\tt //} instead of C style
comments ({\tt /* */}) is recommended to avoid problems arising from nested
comments. Comments should be written in plain english and describe the
functionality of the next few lines.

To check whether a header or source file fulfils this requirement, use the
script {\tt BALL/doc/tools/check\_coding}. Invoke {\tt check\_coding} with the
filename to check (or a list of files) and it will print the percentage of
comments (including Doxygen comments) for each file.

\subsection{Standard header}
                                                                                                                                                             
{\bf BALL uses the Concurrent Version System (CVS)} to manage different
version of the
source files. For easier identification, {\bf each BALL source file starts
with a header containing the CVS ID string in a comment.} Any new BALL header
or source file should start with this header:
\begin{verbatim}
  // -*- Mode: C++; tab-width: 2; -*-
  // vi: set ts=2:
  //
  // $Id: coding_style.tex,v 1.5 2004/02/23 11:09:06 oliver Exp $
  //                                                                                                                                                             
\end{verbatim}
In non-C++ files (Makefiles, TeX-Files, etc.) the C++ comments are replaced
by the respective comment characters (e.g. ``\#'' for Makefiles).

\subsection{Exception handling}

No BALL class method or function should lead to the termination of a program
in case of a detectable error. In particular, if BALL classes are embedded 
for scripting, robustness is an important topic. The recommended
procedure to handle -- even fatal -- errors is to throw an exception. Uncaught
exceptions will result in a call to {\tt abort} thereby terminating the
program while still giving the programmer a chance to handle them graciously.\\
All exceptions used in BALL are derived from {\tt GeneralException}
defined in {\bf COMMON/exception.h}. A default constructor should not be
implemented for these exceptions. Instead, the constructor of all derived
exceptions should have the following signature:
\begin{verbatim}
  AnyException(const char* file, unsigned long line);
\end{verbatim}
Additional arguments are possible but should provide default values (see
{\tt IndexOverflow} for an example).\\
The {\tt throw} directive for each exception should be of the form
\begin{verbatim}
  throw AnyException(__FILE__, __LINE__);
\end{verbatim}
to simplify debugging. {\tt \_\_FILE\_\_} and {\tt \_\_LINE\_\_} are preprocessor
macros filled in by the compiler. {\bf GeneralException} provides two methods
({\tt getFile} and {\tt getLine}) that allow the localization of the
exception's cause.\\


\subsection{Persistence}

Object persistence is a serious problem in C++. There are mainly three
concepts that are responsible for the trouble:
\begin{itemize}
	\item multiple inheritence from a common base class, leading to virtual
		inheritance
	\item static member variables
	\item void pointers
\end{itemize}

To start at the end, {\em void pointers} are easisest to handle: avoid them!
Pointers are only valid inside a single instance of a program, they have no
meaning for a different instance or even another machine. Any code dealing
with void pointers cannot be made persistent.

{\em Static member} variables can cause trouble, if the behaviour of a class depends
on the state of this variable. There is no general rule on how to handle these
variables. However, in most cases everything works fine if you simply ignore
them. The only case where a static member variable had to be considered when
implementing object persistence for the BALL kernel classes was the internal
counter of the class {\tt Object}. This counter is incremented by one for each
instance of {\tt Object} created. It is used to uniquely identify objects and
to define a linear order on these objects (this is the object's handle).
If the state of this counter was written with each persistent object, what
had happened? When writing for example a molecule, the static variable had
always had the same value. Reading these objects again had set the internal
variable of the reading instance of a program to that value. This might lead
to object handles occuring twice. Persistent reading for objects of type {\tt Object} 
was therefore implemented to ignore the value. This leads to a different order
of the handles (the atoms that are read get handles according to the order in
which they are read and created) but prevents the double occurance of the same
handle. Whether static variables have to be read/written or not has to be
decided  on a case-by-case basis.

{\em Multiple inheritance} is the most serious problem. Again, the kernel
classes contain a good example. {\tt Atom} is derived (among others) from
{\tt Composite} and {\tt PropertyManager}. {\tt Composite} is a {\tt
PersistentObject}. Hence, an atom inherits the methods {\tt persistentWrite}
and {\tt persistentRead}. It would be quite easy to implement object
persistence for an atom, if {\tt PropertyManager} was a {\tt
PersistentObejct}, too. This is not possible since {\tt Atom} could not decide
which {\tt persistentWrite} method is meant with {\tt Atom::persistenWrite}:
either {\tt Composite::persistentWrite} or {\tt
PropertyManager::persistentWrite}. 
All persistent object have to be derived from exactly {\em one} persistent
object - in this case {\tt Composite}. The support classes like {\tt
PropertyManager} or {\tt Selectable} are not derived from {\tt
PersistentObject} but are a {\em model of storable}. 
This requires the definition of two methods in the storable class' interface:
{\tt read(PersistenceManager\& pm)} and {\tt write(PersistenceManager\& pm)}.
These methods are quite alike to {\tt persistenWrite} and {\tt
persistentRead} even in the implementation. The difference is that classes
that are a model of storable cannot be read from a persistent stream if they
are not part or base class of a persistent object. This is due to the fact
that the persistence manager needs a well-defined interface to handle the
objects he stores and retrieves and this interface is implemented in the class
{\tt PersistentObject}. 

\section{Testing}

\subsection{General}

Testing is crucial to verify the correctness of the library -- especially when
using C++. But why has it to be {\em so} complicated, using all these macros
and stuff? One of the biggest problems when building large class frameworks is
portability. C++ compilers are strange beasts and there is not a single one
that accepts the same code as any other compiler. Since one of the main
concerns of BALL is portability, we have to ensure that every single line of
code compiles on all platforms. Due to the long compilation times and the
(hopefully in future) large number of different platforms, tests to verify the
correct behaviour of all classes have to be carried out automatically. This
implies a well defined interface for all tests, which is the reason for all
these strange macros. This fixed format also enforces the writing of complete
class tests. Usually a programmer writes a few lines of code to test the parts
of the code he wrote for correctness. Of the methods tested after the
introduction of the test macros, about a tenth of all functions/methods showed
severe errors or after thorough testing. Most of these errors didn't occur an
all platforms or didn't show up on trivial input.

Writing tests for {\em each} method of a class also ensures that each line is
compiled. When using class templates the compiler only compiles the methods
called. Thus it is possible that a code segment contains syntactical errors
but the compiler accepts the code happily -- he simply ignores most of the
code. This is quickly discovered in a complete test of all methods. The same
is true for configuration dependend preprocessor directives that stem from
platform dependencies. Often untested code also hides inside the {\tt const}
version of a method, when there is a non-const method with the same name and
arguments (for example most of the {\tt getName}) methods in BALL. In most
cases, the non-const version is preferred by the compiler and it is usually
not clear to the user which version is taken. Again, explicit testing of each
single method provides help for this problem.
The ideal method to tackle the problem of untested code is the complete
coverage analysis of a class. Unfortunately this is only supported for very
few compilers, so it is not used for testing BALL.

Writing the test program is a wonderful opportunity to verify and complete the
documentation! Often enough implementation details are not clear at the time
the documentation is written. A lot of side effects or special cases that were
added later do not appear in the documentation. Going through the
documentation and the implementation in parallel is the best way to verify the
documentation for consistence and (strange coincidence?!) the best way to
implement a test program, too!


\subsection{Structure of a test program}

Each BALL class has to provide its own test program. This test program has to check
each method of the class. The test programs reside in the directory
{\tt BALL/source/TEST}. To create a new test program, rename a copy of the file
{\tt Skeleton\_test.C} to the new class test name (usually {\tt
<classname>\_test.C}). The test program has to be coded using the class test
macros as described in the BALL online reference (see below). Special care should be taken
to cover all special cases (e.g. what happens, if a method is called with
empty strings, negative values, zero, null pointers etc.). 

\subsection{Macros to start, finish and evaluate tests}
\begin{itemize}
	\item {\tt START\_TEST(class\_name, version)}
	\item {\tt END\_TEST}
	\item {\tt CHECK(name)}
	\item {\tt RESULT}
\end{itemize}

\subsection{Comparison macros}
\begin{itemize}
	\item {\tt TEST\_EQUAL(a, b)}
	\item {\tt TEST\_NOT\_EQUAL(a, b)}
	\item {\tt TEST\_REAL\_EQUAL(a, b)}
	\item {\tt TEST\_EXCEPTION(exception\_type, expression)}	
\end{itemize}



\end{document}
